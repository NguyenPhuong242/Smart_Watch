% ============ DÉVELOPPEMENT ET VALIDATION ============
\chapter{Développement et validation}

\section{Stratégie de test}

La validation de la montre repose sur plusieurs niveaux :
\begin{itemize}
    \item Tests unitaires des drivers Zephyr (capteurs, écran, RTC, BLE) pour vérifier les communications matérielles (I\textsuperscript{2}C, SPI, GPIO).
    \item Tests d'intégration pour valider la cohérence entre machine d'états, interface graphique LVGL et services BLE.
    \item Tests système sur cible réelle (nRF5340 DK + shield IKS01A3 + écran) pour évaluer les performances, la réactivité et la consommation.
\end{itemize}

\section{Cas de test principaux}

Exemples de cas de test :
\begin{itemize}
    \item Vérification de l'affichage correct de l'heure et de la date après configuration de la RTC et après synchronisation BLE.
    \item Test de la détection de mouvements (réveil de l'écran, comptage de pas) et comparaison grossière avec un podomètre de référence.
    \item Test de la boussole : vérification qualitative de la direction affichée par rapport à une boussole physique.
    \item Test des notifications : réception d'un SMS ou d'une notification d'application et affichage sur la montre.
    \item Test des modes basse consommation : mesure approximative de la consommation dans différents états (veille, affichage actif, BLE actif).
\end{itemize}

\section{Critères de validation}

Le projet sera considéré comme validé si :
\begin{itemize}
    \item Les fonctionnalités minimales imposées (Zephyr, LVGL, capteurs, BLE, RTC) sont opérationnelles.
    \item L'interface utilisateur permet une navigation fluide entre les principales applications.
    \item La machine d'états fonctionne sans blocage (pas de deadlock, comportement cohérent).
    \item La consommation reste compatible avec une utilisation sur batterie (ordre de grandeur raisonnable au vu du matériel de développement).
\end{itemize}

\newpage
