% ============ MODÉLISATION DU SYSTÈME ============
\chapter{Modélisation du système}

% ----- Image 1 : 3.1 Bords du modèle -----
\section{Bords du modèle}

Cette section décrit les frontières du système : entrées (capteurs, utilisateur, communication) et sorties (écran, LED, stockage, communication). Le schéma ci-dessous illustre le nRF5340 au centre, connecté à l'écran TFT, au shield IKS01A3, au module RTC et aux interfaces BLE.

\begin{figure}[H]
    \centering
    \IfFileExists{img_modelisation/1_bords_modele.png}{%
        \includegraphics[width=\linewidth]{img_modelisation/1_bords_modele.png}%
    }{\fbox{\parbox{0.8\linewidth}{\centering\vspace{2cm}Image à ajouter : \texttt{1\_bords\_modele.png}\vspace{2cm}}}}
    \caption{Bords du modèle}
    \label{fig:bords_modele}
\end{figure}

% ----- Image 2 : 3.3 Flot de données -----
\section{Flot de données}

Description du flux des données depuis les capteurs et les entrées utilisateur vers le traitement (Zephyr, machine d'états) et les sorties (affichage LVGL, BLE, stockage).

\begin{figure}[H]
    \centering
    \IfFileExists{img_modelisation/2_flot_donnees.png}{%
        \includegraphics[width=\linewidth]{img_modelisation/2_flot_donnees.png}%
    }{\fbox{\parbox{0.8\linewidth}{\centering\vspace{2cm}Image à ajouter : \texttt{2\_flot\_donnees.png}\vspace{2cm}}}}
    \caption{Flot de données}
    \label{fig:flot_donnees}
\end{figure}

% ----- Image 3 : 3.4 Flot d'événements -----
\section{Flot d'événements}

Les événements (tactile, boutons, alarmes RTC, données BLE) déclenchent des traitements et des changements d'état. Le schéma ci-dessous décrit le flot d'événements.

\begin{figure}[H]
    \centering
    \IfFileExists{img_modelisation/3_flot_evenements.png}{%
        \includegraphics[width=\linewidth]{img_modelisation/3_flot_evenements.png}%
    }{\fbox{\parbox{0.8\linewidth}{\centering\vspace{2cm}Image à ajouter : \texttt{3\_flot\_evenements.png}\vspace{2cm}}}}
    \caption{Flot d'événements}
    \label{fig:flot_evenements}
\end{figure}

% ----- Image 4 : Machine d'états -----
\section{Machine d'états}

La machine d'états organise les modes de la montre (veille, heure, menu, activité, synchronisation, configuration), avec une séparation claire entre logique d'application et drivers matériels.

\begin{figure}[H]
    \centering
    \IfFileExists{img_modelisation/4_machine_etats.png}{%
        \includegraphics[width=\linewidth]{img_modelisation/4_machine_etats.png}%
    }{\fbox{\parbox{0.8\linewidth}{\centering\vspace{2cm}Image à ajouter : \texttt{4\_machine\_etats.png}\vspace{2cm}}}}
    \caption{Machine d'états de gestion du système}
    \label{fig:machine_etats}
\end{figure}

\newpage
