% ============ RÉPARTITION DES TÂCHES ============
\chapter{Répartition des tâches}

La répartition des tâches proposée pour un groupe de trois personnes pourrait être la suivante :

\section{Membre A : Plateforme et OS}

\begin{itemize}
    \item Configuration Zephyr (devicetree, overlays, Kconfig) pour le nRF5340 DK, l'écran et le module RTC.
    \item Intégration des drivers de base (display, input, RTC, GPIO).
    \item Mise en place de la machine d'états générale et des threads Zephyr associés.
\end{itemize}

\section{Membre B : Capteurs et BLE}

\begin{itemize}
    \item Intégration du module \texttt{sensor} de Zephyr avec le shield X-NUCLEO-IKS01A3 (LSM6DSO, LIS2MDL, capteurs environnementaux).
    \item Traitement des données (podomètre, boussole, température, etc.).
    \item Mise en œuvre de la communication BLE (services, caractéristiques, protocole d'échange avec le smartphone).
\end{itemize}

\section{Membre C : Interface graphique et UX}

\begin{itemize}
    \item Conception des écrans avec SquareLine Studio (watchfaces, menus, écrans capteurs, réglages).
    \item Intégration LVGL dans Zephyr et gestion des événements tactiles.
    \item Affichage des notifications et des données de capteurs dans l'interface.
\end{itemize}

Une phase commune de tests et d'intégration globale est prévue, impliquant tous les membres.
