% ============ CONCLUSION ============
\chapter{Conclusion}

Ce projet de montre connectée basée sur le nRF5340 et Zephyr RTOS nous a permis d'explorer une architecture complète d'objet connecté, depuis les drivers bas niveau jusqu'à l'interface utilisateur graphique et à la connectivité Bluetooth. En nous inspirant de ZSWatch et en exploitant le shield de capteurs X-NUCLEO-IKS01A3, nous avons pu définir un ensemble de fonctionnalités réalistes, tout en respectant les contraintes de temps et de complexité du module.

La démarche imposée (machine d'états, structuration en modules, utilisation rigoureuse de l'écosystème Zephyr) prépare à la conception de systèmes embarqués plus complexes et constitue une base solide pour de futurs développements dans le domaine des wearables et de l'IoT.
