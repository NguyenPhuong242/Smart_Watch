% ============ INTRODUCTION ============
\chapter{Introduction}

Les objets connectés portables, et en particulier les montres connectées, combinent des contraintes fortes de consommation, d'ergonomie et de connectivité sans fil. Dans ce projet, nous nous inspirons du design ouvert \emph{ZSWatch}, une smartwatch basée sur un SoC Nordic et l'OS temps réel Zephyr, pour concevoir une montre simplifiée sur plateforme nRF5340.

L'objectif du projet est de réaliser une architecture matérielle et logicielle permettant d'exploiter un écran graphique tactile, des capteurs de mouvement et d'environnement (via le shield X-NUCLEO-IKS01A3), une connectivité Bluetooth Low Energy, ainsi qu'un module RTC externe pour la gestion de l'heure et du calendrier.

Nous nous appuierons sur les éléments fournis en TD/TP (devicetree, Kconfig, LVGL, BLE, module \texttt{sensor} de Zephyr) et sur la machine d'états décrite dans le cours de SEC, afin de proposer une solution modulaire et extensible.

\newpage
